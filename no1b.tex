
The regression was re-estimated using a sample of $n = 3500$ observations generated from the same $dgp$:
\[
y_i = 10 + 0.5x_{i2} + 0.5x_{i3} + 0.5x_{i4} + \varepsilon_i, 
\qquad \varepsilon_i \sim N(0,4^2).
\]
The estimated coefficients are $\hat{\beta}_2 = 0.490$, $\hat{\beta}_3 = 0.551$, and $\hat{\beta}_4 = 0.448$, 
which remain very close to the true values $\beta_2 = \beta_3 = \beta_4 = 0.5$.

\begin{enumerate}[label=(\roman*)]
\item 
This similarity is expected, as the OLS estimator is unbiased and consistent:
\[
E[\hat{\beta}] = \beta, \qquad 
\hat{\beta} \xrightarrow{p} \beta \text{ as } n \to \infty.
\]
The variance of the estimator,
\[
\operatorname{Var}(\hat{\beta}\,|\,X) = \sigma^2 (X'X)^{-1},
\]
decreases with $n$ since $X'X$ increases proportionally to the sample size.  Therefore, larger $n$ reduces the sampling variability of $\hat{\beta}$.  The considerably smaller standard errors  
($se(\hat{\beta}_2) = 0.0078$, $se(\hat{\beta}_3) = 0.0685$, and $se(\hat{\beta}_4) = 0.0680$) shows that the collinearity problem for testing decreased, but have not been totally solved. 

\medskip
\textbf{R output:}
\begin{verbatim}
Call:
lm(formula = y ~ x2 + x3 + x4, data = data33)

Residuals:
     Min       1Q   Median       3Q      Max 
-13.2997  -2.5950  -0.0316   2.5728  14.3972 

Coefficients:
             Estimate Std. Error t value Pr(>|t|)    
(Intercept) 10.229438   0.181098  56.486  < 2e-16 ***
x2           0.490229   0.007763  63.151  < 2e-16 ***
x3           0.551196   0.068484   8.048 1.14e-15 ***
x4           0.447534   0.067977   6.584 5.28e-11 ***
---
Signif. codes:  0 ‘***’ 0.001 ‘**’ 0.01 ‘*’ 0.05 ‘.’ 0.1 ‘ ’ 1

Residual standard error: 3.952 on 3496 degrees of freedom
Multiple R-squared: 0.8519,  Adjusted R-squared: 0.8517 
F-statistic: 6702 on 3 and 3496 DF,  p-value: < 2.2e-16
\end{verbatim}

\item

The 95\% confidence intervals were recalculated for the larger sample.  
They are substantially narrower compared with those obtained when $n = 35$.  
For example, the interval for $\beta_2$ changes from $[0.443,\, 0.748]$ to approximately $[0.486,\, 0.514]$.  

\medskip
\noindent
This occurs because the variance of $\hat{\beta}_k$ depends inversely on sample size:
\[
\operatorname{Var}(\hat{\beta}_k \,|\, X) = \sigma^2 [(X'X)^{-1}]_{kk} \propto \frac{1}{n}.
\]
Hence, the standard errors decrease at rate $1/\sqrt{n}$, 
and the corresponding confidence intervals become tighter around the true parameter values.  

\medskip
\noindent
This illustrates the \textbf{consistency} of OLS: as $n$ increases, $\hat{\beta} \xrightarrow{p} \beta$, 
and the confidence intervals converge to the true $\beta$’s.  
The narrowing of the intervals reflects reduced sampling uncertainty and greater estimator precision.

\item

The 95\% confidence region for $(\beta_3, \beta_4)$ becomes smaller when the sample size increases to $n = 3500$.  
The shape of the region (an ellipse) remains the same, but its area decreases as the estimators become more precise.  

\medskip
\noindent
The covariance matrix of the OLS estimator is
\[
\widehat{\operatorname{Var}}(\hat{\beta}) = \hat{\sigma}^2 (X'X)^{-1}.
\]
Since $X'X$ grows proportionally with $n$, $(X'X)^{-1}$ decreases in magnitude, 
leading to smaller variances for $\hat{\beta}_3$ and $\hat{\beta}_4$.  
Consequently, the confidence ellipse contracts around the true parameter values.

\medskip
\noindent
Formally, the 95\% joint confidence region satisfies
\[
(\hat{\beta} - \beta)' 
\big[\widehat{\operatorname{Var}}(\hat{\beta})\big]^{-1} 
(\hat{\beta} - \beta) \leq c_{0.95}.
\]
As $\widehat{\operatorname{Var}}(\hat{\beta})$ decreases with $n$, 
the ellipse becomes smaller.  
This visual reduction in the region reflects higher precision and the asymptotic properties of the OLS estimator.

\item

The reduction in the variance of $\hat{\beta}_3$ (and $\hat{\beta}_4$) when $n$ increases 
can be explained using the variance expression of the OLS estimator:
\[
\operatorname{Var}(\hat{\beta}_j \mid X) = \sigma^2 \big[(X'X)^{-1}\big]_{jj}.
\]
Because $X'X = \sum_{i=1}^n x_i x_i'$ scales with $n$, 
its inverse $(X'X)^{-1}$ decreases approximately at rate $1/n$.  
Therefore, for a given $\sigma^2$,
\[
\operatorname{Var}(\hat{\beta}_j \mid X) \propto \frac{\sigma^2}{n}.
\]
This implies that as $n$ increases, the sampling variability of $\hat{\beta}_3$ and $\hat{\beta}_4$ decreases, 
producing smaller standard errors and tighter confidence regions.

\medskip
\noindent
Intuitively, a larger sample provides more information about the relationship between $y$ and the regressors, 
which improves the precision of the estimated slope coefficients.  
Hence, the observed reduction in the size of the confidence region directly follows from the 
$1/n$ relationship in the variance expression.



\end{enumerate}